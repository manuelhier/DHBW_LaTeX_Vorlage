%%%%%%%%%%%%%%%%%%%%%%%%%%%%%%%%%%%%%%%%%%%%%%%%%%%%%%%%%%%%%%%%%%%%%%%%%%%%%%%
%% 2020-06-20
%% Descr:       Formale Einstellungen für die Praxisarbeit
%% Author:      Vorlage erstellt von Daniel Spitzer an der DHBW Lörrach
%% Angepasst:   Manuel Rettig
%%%%%%%%%%%%%%%%%%%%%%%%%%%%%%%%%%%%%%%%%%%%%%%%%%%%%%%%%%%%%%%%%%%%%%%%%%%%%%%

\documentclass[a4paper,12pt]{article}

% Sprache
\usepackage[english, ngerman]{babel} % Hier ggf. Sprache anpassen
% \usepackage{polyglossia}           % Babel Ersatz
% \setdefaultlanguage[spelling=new, babelshorthands=true]{german}

% Schriftart
\usepackage{lmodern}
\usepackage{fontspec}
% \usepackage{carlito}
% \setmainfont{TeX Gyre Pagella}    % Serif
% \setmainfont{TeX Gyre Termes}     $ Serif
% \setsansfont{Arial}
% \setmonofont{Courier New}
% \setmainfont{TeX Gyre Heros}      % Sans serif
% \setmainfont{carlito}             % Sans serif

% Seitenränder Projekt bzw. Bachlorarbeit
\usepackage[left=3.5cm, right=2.5cm, bottom=2cm, top=2.5cm, head=1.25cm, foot=1.25cm, includefoot]{geometry}

% Seitenränder Links / Rechts Symmetrisch z.B. für Portfolio oder Assignments
% \usepackage[left=2.5cm, right=2.5cm, head=1.25cm, bottom=2cm, foot=1.25cm, includefoot]{geometry}

% Zeilenabstand: 1.5
\usepackage{setspace}
\setstretch{1.5}

% Inhaltsverzeichnis
\usepackage[nottoc]{tocbibind}
\usepackage{parskip}

% Abkürzungsverzeichnis
\usepackage{acro}
\DeclareAcronym{be}{
    short = BE,
    long  = Business Engineering
}

\DeclareAcronym{ki}{
    short = KI,
    long  = Künstliche Intelligenz
}

\DeclareAcronym{gki}{
    short = GenKI,
    long  = Generative Künstliche Intelligenz
}

\DeclareAcronym{llm}{
    short = LLM,
    long  = Large Language Model
}

\DeclareAcronym{pjm}{
    short = PJM,
    long  = Projektmanagement
}

\DeclareAcronym{pm}{
    short = PM,
    long  = Produktmanagement
}

\DeclareAcronym{tpm}{
    short = TPM,
    long  = Technisches Produktmanagement
}

\DeclareAcronym{re}{
    short = RE,
    long  = Requirements Engineering
}

\DeclareAcronym{rag}{
    short = RAG,
    long  = Retrieval-Augmented Generation
}
\acsetup{
  list/name = \acronymHeading,
  list/preamble = \addcontentsline{toc}{section}{\acronymHeading},
  list/display=all
}

% Literaturverzeichnis & Zitieren
\usepackage[backend=biber, style=apa, language=german, date=year, isbn=false, eprint=false, url=false, maxnames=2, minnames=1]{biblatex}  % APA
% \usepackage[backend=biber, style=alphabetic, maxalphanames=1, language=german, date=year, isbn=false, eprint=false, url=false, maxnames=2, minnames=1]{biblatex}  % Alphabetic

\DefineBibliographyStrings{german}{andothers={et\addabbrvspace al\adddot}}

\addbibresource{./text/bibliography.bib}
\renewcommand*{\labelalphaothers}{}
\newcommand{\textciteyear}[1]{\textcolor{darkgray}{\citeauthor{#1} \cite{#1}}}

% --- \cite in Grau + eckigen Klammern ---
% \DeclareCiteCommand{\cite}
%   {\usebibmacro{prenote}}
%   {\textcolor{darkgray}{\mkbibbrackets{\usebibmacro{citeindex}\usebibmacro{cite}}}}
%   {\multicitedelim}
%   {\usebibmacro{postnote}}

% --- \textcite in Grau + Autor (Jahr) ---
\DeclareCiteCommand{\textcite}
{\usebibmacro{prenote}}
{\textcolor{darkgray}{\printnames{labelname}%
\addspace\mkbibparens{\printfield{year}}}}
{\multicitedelim}
{\iffieldundef{postnote}{}{\addcomma\space\printfield{postnote}}}

% Fuß und Kopfzeilen
\usepackage{fancyhdr}
\renewcommand{\headrulewidth}{0pt}
\renewcommand{\footrulewidth}{0pt}
\def \footer{
  \begin{center}
    \hfill
    %\fontsize{\footerFontSize}{\footerFontSize}\selectfont\thesisFooterTitle
    \hfill
    \thepage
  \end{center}
}
\fancyhead{}
\fancyfoot{}
\fancyfoot[R]{\footer}

% Positioniert die Fußnoten fest am unteren Ende der Seite
\usepackage[hang, flushmargin, bottom, multiple]{footmisc}
\renewcommand{\hangfootparindent}{0.5em}
\renewcommand{\footnotelayout}{\hspace{0.5em}}

% Tabellen & Abbildungen
% \usepackage{tabulary}
\usepackage{tabularx}
\usepackage{makecell}
\usepackage{rotating} % in der Präambel
\usepackage[table]{xcolor}

\definecolor{lightgray}{gray}{0.98}
\definecolor{darkgrey}{gray}{0.7}
\definecolor{color_strength}{HTML}{D8EFCB}   % sehr helles Grün
\definecolor{color_weakness}{HTML}{FFE8CC} % sehr helles Orange
\definecolor{color_opportunity}{HTML}{D6E5FF}    % sehr helles Blau
\definecolor{color_threat}{HTML}{FFD6D6}    % sehr helles Rot

\usepackage{float}  % Positioniert Tabellen und Abbildungen
\usepackage{graphicx}
\graphicspath{{./config}{./text/images/}{./text/appendix/}}
\usepackage[labelfont=bf]{caption}
\captionsetup{belowskip=-9pt}
\usepackage{ragged2e}

\usepackage{pdflscape}
\usepackage{booktabs}
\usepackage{array}

% X -> p{} (obenbündig), gut für Listen & Blocksatz
\newenvironment{TabXtop}{%
  \begingroup
  \renewcommand\tabularxcolumn[1]{>{\arraybackslash}p{##1}}%
}{\endgroup}

% X -> m{} (vertikal mittig), gut für Matrix
\newenvironment{TabXmid}{%
  \begingroup
  \renewcommand\tabularxcolumn[1]{m{##1}}%
}{\endgroup}

\newcolumntype{P}[1]{>{\raggedright\arraybackslash}p{#1}}
\newcolumntype{R}[1]{>{\raggedleft\arraybackslash}p{#1}}

\newcolumntype{C}{>{\centering\arraybackslash}X}
\newcolumntype{L}{>{\RaggedRight\arraybackslash}X}

% X-Spalten m-ähnlich machen (vertikale Mittelung wie m{})
\renewcommand\tabularxcolumn[1]{>{\arraybackslash}p{#1}}

\usepackage{tcolorbox}
\tcbset{colback=gray!5,colframe=gray!40!black,boxrule=0.5pt,arc=2mm}

% Aufzählungen
\usepackage{enumitem}
\setlist[description]{itemsep=0pt, parsep=0pt, topsep=0pt}
\setlist{topsep=\parskip, itemsep=0pt, parsep=0pt}

% Code-Blöcke
\usepackage{minted}

% Mathematik
\usepackage{mathtools}
\usepackage{amssymb}

% Anführungszeichen mit \enquote{...}
\usepackage{csquotes}
\newcommand{\epigraph}[2]{ % Befehl für ein einleitendes Zitat
  \begin{quote}
    \begin{quote}
      \begin{center}
        \textit{#1}
      \end{center}
      \hfill #2
    \end{quote}
  \end{quote}
}

\usepackage{pdfpages}
\newcommand{\interviewref}[1]{%
  % \hyperref[app:interview#1]{E#1}%
  {E#1}%
}

% Das Paket hyperref muss als letztes Paket geladen werden. Das Paket setzt Links im PDF-Dokument.
\usepackage[hidelinks, unicode]{hyperref}
\hypersetup{pdftitle = {\thesisFooterTitle}, pdfauthor = {\name}}

% Normale Schriftart für URLs
\renewcommand{\UrlFont}{}

% Nummerierung: 2.1, 2.2 usw.
\renewcommand{\labelenumii}{\theenumii}
\renewcommand{\theenumii}{\theenumi.\arabic{enumii}.}
